\documentclass[a4paper,12pt]{article}
\usepackage[MeX]{polski}
\usepackage[utf8]{inputenc}

%opening
\title{Wydział Matematyki i Informatyki Uniwersytetu
Warmińsko-Mazurskiego}
\author{Marcel Mendziński}

\begin{document}

\maketitle  
\tableofcontents
\begin{abstract}
	 Wydział Matematyki i Informatyki Uniwersytetu Warmińsko-Mazurskiego (WMiI) – wydział
Uniwersytetu Warmińsko-Mazurskiegow Olsztynie oferujący studia na dwóch kierunkach:
	
	\begin{itemize}
		\item Informatyka
		\item Matematyka
	\end{itemize}
	
w trybie studiów stacjonarnych i niestacjonarnych. Ponadto oferuje studia podyplomowe.
Wydział zatrudnia 8 profesorów, 14 doktorów habilitowanych, 53 doktorów i 28 magistrów.
\end{abstract}

\section{Misja}
	Misją Wydziału jest:
\begin{itemize}
	\item Kształcenie matematyków zdolnych do udziału w rozwijaniu matematyki i jej stosowania w innych działach wiedzy i w praktyce;
	\item Kształcenie nauczycieli matematyki, nauczycieli matematyki z fizyką a także nauczycieli informatyki;
	\item Kształcenie profesjonalnych informatyków dla potrzeb gospodarki, administracji, szkolnictwa oraz życiaspołecznego;
	\item Nauczanie matematyki i jej działów specjalnych jak statystyka matematyczna, ekonometria, biomatematyka, ekologia matematyczna, metody numeryczne; fizyki a w razie potrzeby i podstaw informatyki na wszystkich wydziałach UWM.
\end{itemize}
	
\section{Opis kierunków}

	Na kierunku Informatyka prowadzone są studia stacjonarne i niestacjonarne:
	\begin{itemize}
		\item studia pierwszego stopnia – inżynierskie (7 sem.), sp. inżynieria systemów informatycznych, informatyka ogólna
		\item studia drugiego stopnia – magisterskie (4 sem.), sp. techniki multimedialne, projektowanie systemów informatycznych i sieci komputerowych
	\end{itemize}
	
	
	
\end{document}