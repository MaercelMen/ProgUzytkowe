\documentclass[]{beamer}
%\usepackage[MeX]{polski}
%\usepackage[cp1250]{inputenc}
\usepackage{polski}
\usepackage[utf8]{inputenc}
\beamersetaveragebackground{blue!10}
\usetheme{Warsaw}
\usecolortheme[rgb={0.1,0.5,0.7}]{structure}
\usepackage{beamerthemesplit}
\usepackage{multirow}
\usepackage{multicol}
\usepackage{array}
\usepackage{graphicx}
\usepackage{enumerate}
\usepackage{amsmath} %pakiet matematyczny
\usepackage{amssymb} %pakiet dodatkowych symboli
\usepackage{hyperref}
\title{Układ klawiatury}
\author{Marcel Mendziński}
\date{\today}
\begin{document}

\frame
{
\maketitle
}

\begin{frame}
	\frametitle{Spis Treści}
	\tableofcontents

\end{frame}

\begin{frame}{Układ klawiatury}
	\section{Układ klawiatury}
	Układ klawiatury – ułożenie znaków literowych, cyfr i innych oraz pozostałych przycisków 	funkcyjnych na dowolnym urzdeniu, z którym powiązany jest sposób uzyskiwania znaków diakrytycznych.
\end{frame}

\begin{frame}{Wygl\k{a}d klawiatury}
	\section{Wygl\k{a}d klawiatury}
	Klawiatura występuje zazwyczaj w trzech odmianach: ANSI (ANSI-INCITS 154-1988) szeroki enter i klawisz z ``{\textbar}'' i ``$\backslash$", ISO (ISO/IEC 9995-2) wysoki enter, JIS (JIS X 6002-1980) wysoki enter i krótki backspace.
\end{frame}

\begin{frame}{Rodzaje Klawiatury}
 \section{Rodzaje Klawiatury}
		Wyrużniamy różne rodzaje klawiatór :
		\begin{itemize}
			\item \href{https://pl.wikipedia.org/wiki/AZERTY}{AZERTY}
			\item \href{https://pl.wikipedia.org/wiki/Klawiatura_Dvoraka}{klawiatura Dvoraka}
			\item \href{https://pl.wikipedia.org/wiki/Klawiatura_maszynistki}{klawiatura maszynistki}
			\item \href{https://pl.wikipedia.org/wiki/QWERTY}{QWERTY}
			\item \href{https://pl.wikipedia.org/wiki/QWERTZ}{QWERTZ}
		\end{itemize}
\end{frame}
\

\end{document}

